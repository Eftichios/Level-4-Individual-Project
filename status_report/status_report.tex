    
\documentclass[11pt]{article}
\usepackage{times}
    \usepackage{fullpage}
    
    \title{ Distributed game using adverts and trackers in web browsers}
    \author{ Eftychios Karagiorgis - 2329664k }

    \begin{document}
    \maketitle
    
    
     

\section{Status report}

\subsection{Proposal}\label{proposal}

\subsubsection{Motivation}\label{motivation}

Ad targeting ecosystems are complex, the lack of transparency in terms of what exact data is used and how coupled with the lack of controls that users have causes concerns over privacy of personal data. Advertisers usually profile users based on personal data such as their browsing history, social media profiles and other data gathered from tracking (or otherwise) in order to deliver targeted ads. Putting the user's knowledge of these systems to the test could provide useful insights as to how they think the profiling is done. 

\subsubsection{Aims}\label{aims}

The aim is to develop a multiplayer game using adverts and trackers as the main game resources to explore user's understanding of targeted advertising as well as raise awareness of the various issues that come with it through game mechanics and information sections. An idea for the game-play is: in a game session, players will be given an advert category and then attempt to trick the profiling done by advertisers in order to receive adverts in that category. This will be implemented as a web application along with a browser extension. The web app will allow users to find games as well as have game metrics and achievements while the extension will gather advert information during the game-play. The behaviour of the users will be evaluated by analysing the data gathered from the game-play. Furthermore, the enjoyment and engagement of the game will be evaluated through surveys.

\subsection{Progress}\label{progress}

\begin{itemize}
\item  Technologies and frameworks chosen: front-end will be implemented using React, back-end using Node.js with express. The database will be a postgresql database, we will use Sequelize (object-relational mapper) to communicate with the database rather than raw sql. The extension will be implemented in chrome. WebSockets will be used to allow for real-time communication between different parties which will be implemented using Socket.io.
\item  Background research done on the infrastructure of targeted advertising as well as different privacy issues involved with it.
\item  Exploration of different technologies that identify tracking such as Ad Blockers and Blacklight.
\item  Initial design prototypes and refinements using surveys.
\item  Initial software architecture and React component diagram.
\item  Web app and extension wireframes using Figma.
\item  ER diagram and design of the database.
\item  Basic web application developed with authentication and authorisation.
\item  Chrome extension developed to observe and count ad trackers with a corresponding interface that displays relevant information.
\item  Database creation and population with dummy data, automated using a script with Sequelize for object-relational mapping.
\item  Implemented REST API for all table data in the database.
\item  Implemented a lobby system that puts players in appropriate lobbies before a game.
\item  Implemented multi-way communication between server, extension and front-end.
\item  Initial version of a multiplayer game-play that counts ad trackers, the players need to get tracked by a certain number of ad trackers to win.

\end{itemize}

\subsection{Problems and risks}\label{problems-and-risks}

\subsubsection{Problems}\label{problems}

\begin{itemize}
\item  Initially the idea was to use Facebook as the playground using dummy accounts, however, due to Facebook's policies on fake accounts and bot-like behaviour, dummy accounts were being banned on creation. Since using the user's personal account could compromise their privacy, we have decided to use tracking in web browsers instead. In chrome, players can create new profiles very easily with fresh browsing history and no associated email.
\item  Locating and classifying adverts on a page is very challenging as there is no descriptive labels. The best approach seems to be: getting the URL of the website the advert redirects to and using a URL classifier to return the most prominent category on that website. However, we have not yet found an open-source URL classification API. Other game modes are already being explored.
\item  The way that ad trackers are identified is by matching them with an open-source list of known ad tracking domains. Further information on ad-trackers can be found in DuckDuckGo's tracker radar but in some cases it is very vague. Additionally, there might be some rare cases were a player encounters ad trackers that there is no information for, meaning that they will not be identified during the game-play.
\end{itemize}

\subsubsection{Risks}\label{risks}

\begin{itemize}
\item  In a survey, we presented participants with three variations of the game-play using as much information on ad trackers as possible. In almost all cases, participants stated that the game-play will get boring really quick as it is repetitive and there is no clear goal other than achievements. There is not something that can be done about this as the setup of the game-play is meant to be educational and experimental and the resources we have to build the game-play are very limited.
\item  The extension will only be available in chrome as it is the most used web browser. This means that players must use chrome to play the game which narrows the pool of participants.

\end{itemize}

\subsection{Plan}\label{plan}

Below is the plan for the second semester. Alongside the tasks outlined, I will be continuously writing the dissertation.
\begin{itemize}
\item  \textbf{Week 1-2:} Add functionality to all pages in the web application (leader-boards, game history, profile editing, etc.) and decide on a final game mode. Add unit testing for all functional requirements. \textbf{Deliverable:} A fully functional and bug-free web application and a plan for a game-play.
\item  \textbf{Week 3-4:} Finalise game-play, build an achievement system and deploy web app. \textbf{Deliverable:} Players will be able to play games and earn achievements.
\item  \textbf{Week 5:} First run of game-play experiments and evaluation. \textbf{Deliverable:} An evaluation of what users think about the game and analysis of their game-play. 
\item  \textbf{Week 6-7:} Refine the web app based on evaluation and run second round of experiments. \textbf{Deliverable:} An improved version of the web-app and game-play as well as further evaluation.
\item  \textbf{Week 8:} Finalise all aspects of the project and deploy final version of web app. Include documentation and information for players to help them set up the extension. \textbf{Deliverable:} The complete product deployed.
\item  \textbf{Week 9-10:} Finish writing the dissertation and submit project before deadline.
\end{itemize}

    
\subsection{Ethics and data}\label{ethics}
This project will use participants for surveys as well as for testing and evaluating the game-play. Surveys will be used to gather ideas and opinions on design choices in order to improve and refine the design of the web application in stages. To evaluate the game-play, participants will play the game and they will be asked to answer a survey to capture user impression and opinions on the actual game-play. Additionally, during the game-play, relevant metrics will be gathered (such as what websites were visited during the game) to allow us to explore the participant's behaviour. \
None of the data gathered from the surveys or the game-play will be personally identifiable information, the participants will play the game under an alias and surveys will be anonymous. Therefore, the Ethics Checklist is appropriate for the nature of the experiments in this project.


\end{document}
